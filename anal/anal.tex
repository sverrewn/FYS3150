%% USEFUL LINKS:
%% -------------
%%
%% - UiO LaTeX guides:          https://www.mn.uio.no/ifi/tjenester/it/hjelp/latex/
%% - Mathematics:               https://en.wikibooks.org/wiki/LaTeX/Mathematics
%% - Physics:                   https://ctan.uib.no/macros/latex/contrib/physics/physics.pdf
%% - Basics of Tikz:            https://en.wikibooks.org/wiki/LaTeX/PGF/Tikz
%% - All the colors!            https://en.wikibooks.org/wiki/LaTeX/Colors
%% - How to make tables:        https://en.wikibooks.org/wiki/LaTeX/Tables
%% - Code listing styles:       https://en.wikibooks.org/wiki/LaTeX/Source_Code_Listings
%% - \includegraphics           https://en.wikibooks.org/wiki/LaTeX/Importing_Graphics
%% - Learn more about figures:  https://en.wikibooks.org/wiki/LaTeX/Floats,_Figures_and_Captions
%% - Automagic bibliography:    https://en.wikibooks.org/wiki/LaTeX/Bibliography_Management  (this one is kinda difficult the first time)
%%
%%                              (This document is of class "revtex4-1", the REVTeX Guide explains how the class works)
%%   REVTeX Guide:              http://www.physics.csbsju.edu/370/papers/Journal_Style_Manuals/auguide4-1.pdf
%%
%% COMPILING THE .pdf FILE IN THE LINUX IN THE TERMINAL
%% ----------------------------------------------------
%%
%% [terminal]$ pdflatex report_example.tex
%%
%% Run the command twice, always.
%%
%% When using references, footnotes, etc. you should run the following chain of commands:
%%
%% [terminal]$ pdflatex report_example.tex
%% [terminal]$ bibtex report_example
%% [terminal]$ pdflatex report_example.tex
%% [terminal]$ pdflatex report_example.tex
%%
%% This series of commands can of course be gathered into a single-line command:
%% [terminal]$ pdflatex report_example.tex && bibtex report_example.aux && pdflatex report_example.tex && pdflatex report_example.tex
%%
%% ----------------------------------------------------


\documentclass[english,notitlepage,reprint,nofootinbib]{revtex4-1}  % defines the basic parameters of the document
% For preview: skriv i terminal: latexmk -pdf -pvc filnavn
% If you want a single-column, remove "reprint"

% Allows special characters (including æøå)
\usepackage[utf8]{inputenc}
% \usepackage[english]{babel}

%% Note that you may need to download some of these packages manually, it depends on your setup.
%% I recommend downloading TeXMaker, because it includes a large library of the most common packages.

\usepackage{physics,amssymb}  % mathematical symbols (physics imports amsmath)
\include{amsmath}
\usepackage{graphicx}         % include graphics such as plots
\usepackage{xcolor}           % set colors
\usepackage{hyperref}         % automagic cross-referencing
\usepackage{listings}         % display code
\usepackage{subfigure}        % imports a lot of cool and useful figure commands
% \usepackage{float}
%\usepackage[section]{placeins}
\usepackage{algorithm}
\usepackage[noend]{algpseudocode}
\usepackage{subfigure}
\usepackage{tikz}
\usetikzlibrary{quantikz}
% defines the color of hyperref objects
% Blending two colors:  blue!80!black  =  80% blue and 20% black
\hypersetup{ % this is just my personal choice, feel free to change things
	colorlinks,
	linkcolor={red!50!black},
	citecolor={blue!50!black},
	urlcolor={blue!80!black}}





\begin{document}
% ===========================================
\section{the equations of motion}

\begin{equation*}
    m \ddot{v} = \sum_i \vectorbold{F}_i
\end{equation*}

This gives us:
\begin{equation*}
    \Rightarrow m 
    \begin{bmatrix}
        \vectorbold{\ddot{x}} \\
        \vectorbold{\ddot{y}} \\
        \ddot{z}
    \end{bmatrix}
    = \sum_i
    \begin{bmatrix}
        \vectorbold{F}_x \\
        \vectorbold{F}_y \\
        F_z
    \end{bmatrix}
    = \sum_i
\end{equation*}

\begin{equation*}
    m \ddot{x} = \sum_i F_{x,i}
\end{equation*}

Simple particle:
\begin{equation*}
    \Rightarrow
    \begin{bmatrix}
        \dot{x} \\
        \dot{y} \\
        0
    \end{bmatrix}
    \times
    \begin{bmatrix}
        0\\
        0 \\
        B_0
    \end{bmatrix}
    =
    \begin{bmatrix}
        B_0 \dot{y} \\
        B_0 \dot{x} \\
        0
    \end{bmatrix}
\end{equation*}

\begin{align*}
    m \ddot{x} &= F_x \\
    &= q E_x + (q \vectorbold{V} \times B)_x \\
    &= -q \frac{\partial v}{\partial x} + q
    \begin{bmatrix}
        \dot{x} \\
        \dot{y} \\
        0
    \end{bmatrix}
    \times
    \begin{bmatrix}
        0 \\
        0 \\
        B_0
    \end{bmatrix}
    &= - q \left( - \frac{V_0}{d^2} x \right) + q B_0 \ddot{y} \\
\end{align*}

\begin{equation*}
    \ddot{x} - \frac{q B_0}{m} \dot{y} - \frac{q V_0}{m d^2} x = 0
\end{equation*}

\begin{equation*}
    \Rightarrow \ddot{x} - \omega_0 \dot{y} - \frac{1}{2} \omega^2_z x = 0
\end{equation*}

\begin{align*}
    m \ddot{y} &= F_y \\
    &= q E_y + (q \vectorbold{V} \times B)_y \\
    &= - q \frac{\partial v}{\partial y} - q B_o \dot{x}
\end{align*}

\begin{equation*}
    \Rightarrow \ddot{y} + \frac{q B_0}{m} \dot{x} - \frac{q V_0}{m d^2} = 0
\end{equation*}

\begin{equation*}
    m \ddot{z} = q E_z = -q \frac{\partial v}{\partial z} = - \frac{4 q v_0}{2 d^2} z
\end{equation*}

\begin{equation*}
    \Rightarrow \ddot{x} + w^2_z = 0
\end{equation*}

\begin{equation*}
    z = A \cos (\omega_z z) + B \sin (\omega_z z)  \;\;\; \square
\end{equation*}


\section{Single differential equation}

\begin{equation*}
    m \ddot{x} - \omega_0 \dot{y} - \frac{1}{2} w_z^2 x + i(\ddot{y} + \omega_z^2) = 0
\end{equation*}

\begin{equation*}
    \Rightarrow \ddot{x} + i \ddot{y} + i w_0 dot{x} - \omega_0 \dot{y} - \frac{1}{2} \omega_z^2 (x + iy) = 0
\end{equation*}

\begin{equation*}
    \Rightarrow \ddot{f} + i \omega_0 (\dot{x} + i \dot{y}) - \frac{1}{2} \omega_z^2 f = 0
\end{equation*}

\begin{equation*}
    \Rightarrow \ddot{f} + i \omega_0 \dot{f} - \frac{1}{2} \omega_z^2 f = 0 \;\;\; \square
\end{equation*}


\section{Necessary constraint on $\omega_0$ and $\omega_z$}

Assuming $\omega \pm \in \mathbb{R}$:
\begin{align*}
    &\Rightarrow \omega_0^2 - 2 \omega_z^2 > 0\\
    &\Rightarrow \omega_0^2 > 2 \omega z^2\\
    &\Rightarrow \left( \frac{q B_0}{m} \right) ^2 > \left( \frac{2 q v_0}{m d^2} \right)\\
    &\Rightarrow \frac{q}{m} B_0^2 > \frac{4 v_0}{d^2} \;\;\; \square
\end{align*}



\section{Lower bounds}

\begin{equation*}
    A_+ (\cos \omega_+ t + i \sin \omega_+ t) + A_- (\cos \omega_- t + i \sin \omega_- t)
\end{equation*}

\begin{equation*}
    \Re \left( f(t) \right) + A_+ \cos \omega_+ t + A_- \cos \omega_- t
\end{equation*}

\begin{equation*}
    \Im \left( f(t) \right) + A_+ \sin \omega_+ t + A_- \sin \omega_- t
\end{equation*}

We are getting upper bounds by when $x(t)$ and $y(t)$ are in upper phase:
\begin{equation*}
    R_+ = A_+ + A_-
\end{equation*}

and lower bounds when they are in opposite phase:
\begin{equation*}
    R_- = | A_+ + A_- |
\end{equation*}


\section{Specific solution of $z(t)$}

\begin{equation*}
    x(0) = \Re \left( f(0) \right) = A_+ + A_- = x_0
\end{equation*}

\begin{equation*}
    y(0) = \Im \left( f(0) \right) = 0
\end{equation*}

\begin{align*}
    \dot{f} &= \frac{d}{dt} \left[ A_+ e^{- \omega_+ t} + A_- e^{- \omega_- t}  \right] \\
    &= -A_+ i \omega_+ e^{-i \omega t} - A_- i \omega e^{-i \omega t}
\end{align*}

\begin{equation*}
    \dot{y} = \Im( \dot{f} ) = -A_+ \omega_+ \cos \omega_+ t - A_- \omega_- \cos \omega_- t
\end{equation*}

\begin{equation*}
    \dot{y}(0) = v_0
\end{equation*}

\begin{equation*}
    \Rightarrow A_+ \omega_+ - A_- \omega_- = v_0
\end{equation*}

\begin{align*}
    \dot{x} &= \Re ( \dot{f} )\\
    &= A_+ \omega_+ \sin \omega_+ t + A_- \omega_- \sin \omega_- t \\
    &= 0
\end{align*}

\begin{equation} \label{eq:i}
    x(0) = x_0 \Rightarrow A_+ + A_- = x_0
\end{equation}

\begin{equation} \label{eq:ii}
    y(0) = v_0 \Rightarrow A_+ \omega_+ - A_- \omega_- = v_0
\end{equation}

\ref{eq:i} gives us 

\begin{equation} \label{eq:iii}
    A_- = x_0 - A_+
\end{equation}

Sets \ref{eq:iii} in to \ref{eq:ii}:
\begin{align} \label{eq:iv}
    &\Rightarrow -A_+ \omega_+ - x_0 \omega_- + A_+ \omega_- = v_0 \nonumber \\
    &\Rightarrow A_+ ({omega_- \omega_+}) = v_0 + x_0 \omega_- \nonumber \\
    &\Rightarrow A_+ = \frac{v_0 + x_0 \omega_-}{\omega_- - \omega_+}
\end{align}

Sets \ref{eq:iv} in to \ref{eq:iii}:
\begin{align*}
    A_- &= x_0 - \frac{v_0 + x_0 \omega_-}{\omega_- - \omega_+} \\
    &= \frac{x_0 (\omega_- - \omega_+) - v_0 - x_0 \omega_-}{\omega_- - \omega_+} \\
    &= \frac{-x_0 \omega_+ - v_0}{\omega_- - \omega_+} \\
    &= - \frac{v_0 + x_0 \omega_+}{\omega_- - \omega_+}
\end{align*}

\begin{equation*}
    z(0) = z_0 \Rightarrow A \cos \omega_z \theta + B \sin \omega_z \theta = z_0 \Rightarrow A = z_0
\end{equation*}

\begin{equation*}
    \dot{z} = \omega_z (-z_0 \sin \omega_z t + V \cos \omega_z t)
\end{equation*}

\begin{equation*}
    \dot{z}(0) = \omega_z (- z_0 \sin \omega_z \dot 0 + B \cos \omega_z \theta) = w_z B = 0 \Rightarrow B = 0
\end{equation*}

\begin{equation*}
    z(t) = z_0 \cos (\omega_z t) \;\;\; \square
\end{equation*}

% ===========================================
\end{document}