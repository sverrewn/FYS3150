%% USEFUL LINKS:
%% -------------
%%
%% - UiO LaTeX guides:          https://www.mn.uio.no/ifi/tjenester/it/hjelp/latex/
%% - Mathematics:               https://en.wikibooks.org/wiki/LaTeX/Mathematics
%% - Physics:                   https://ctan.uib.no/macros/latex/contrib/physics/physics.pdf
%% - Basics of Tikz:            https://en.wikibooks.org/wiki/LaTeX/PGF/Tikz
%% - All the colors!            https://en.wikibooks.org/wiki/LaTeX/Colors
%% - How to make tables:        https://en.wikibooks.org/wiki/LaTeX/Tables
%% - Code listing styles:       https://en.wikibooks.org/wiki/LaTeX/Source_Code_Listings
%% - \includegraphics           https://en.wikibooks.org/wiki/LaTeX/Importing_Graphics
%% - Learn more about figures:  https://en.wikibooks.org/wiki/LaTeX/Floats,_Figures_and_Captions
%% - Automagic bibliography:    https://en.wikibooks.org/wiki/LaTeX/Bibliography_Management  (this one is kinda difficult the first time)
%%
%%                              (This document is of class "revtex4-1", the REVTeX Guide explains how the class works)
%%   REVTeX Guide:              http://www.physics.csbsju.edu/370/papers/Journal_Style_Manuals/auguide4-1.pdf
%%
%% COMPILING THE .pdf FILE IN THE LINUX IN THE TERMINAL
%% ----------------------------------------------------
%%
%% [terminal]$ pdflatex report_example.tex
%% 
%% Run the command twice, always.
%%
%% When using references, footnotes, etc. you should run the following chain of commands:
%%
%% [terminal]$ pdflatex report_example.tex
%% [terminal]$ bibtex report_example
%% [terminal]$ pdflatex report_example.tex
%% [terminal]$ pdflatex report_example.tex
%%
%% This series of commands can of course be gathered into a single-line command:
%% [terminal]$ pdflatex report_example.tex && bibtex report_example.aux && pdflatex report_example.tex && pdflatex report_example.tex
%%
%% ----------------------------------------------------


\documentclass[english,notitlepage,reprint,nofootinbib]{revtex4-1}  % defines the basic parameters of the document
% For preview: skriv i terminal: latexmk -pdf -pvc filnavn
% If you want a single-column, remove "reprint" 

% Allows special characters (including æøå)
\usepackage[utf8]{inputenc}
% \usepackage[english]{babel}

%% Note that you may need to download some of these packages manually, it depends on your setup.
%% I recommend downloading TeXMaker, because it includes a large library of the most common packages.

\usepackage{physics,amssymb}  % mathematical symbols (physics imports amsmath)
\include{amsmath}
\usepackage{graphicx}         % include graphics such as plots
\usepackage{xcolor}           % set colors
\usepackage{hyperref}         % automagic cross-referencing
\usepackage{listings}         % display code
\usepackage{subfigure}        % imports a lot of cool and useful figure commands
% \usepackage{float}
%\usepackage[section]{placeins}
\usepackage{algorithm}
\usepackage[noend]{algpseudocode}
\usepackage{subfigure}
\usepackage{tikz}
\usetikzlibrary{quantikz}
% defines the color of hyperref objects
% Blending two colors:  blue!80!black  =  80% blue and 20% black
\hypersetup{ % this is just my personal choice, feel free to change things
    colorlinks,
    linkcolor={red!50!black},
    citecolor={blue!50!black},
    urlcolor={blue!80!black}}


% ===========================================


\begin{document}

\title{2D Ising model}  % self-explanatory
\author{Sverre Wehn Noremsaune, Jon Aleksander Prøitz, Marius Torsvoll, Frida Marie Engøy Westby} % self-explanatory
\date{\today}                             % self-explanatory
\noaffiliation                            % ignore this, but keep it.

%This is how we create an abstract section.
\begin{abstract}
    insert abstract
\end{abstract}
\maketitle


% ===========================================
\section{Introduction}
%

% ===========================================
\section{Methods}\label{sec:methods}
%
For the Ising model we have studied the 2D square lattices of spins $s_i$. The spin configuration can be up (+1) or down (-1) (as seen in TABLE \ref{tab:possible_states}).

The model uses a lattice with periodic boundary conditions, which have length $L$ and $N$ spins. This gives us

\begin{equation}
    N = L^2
\end{equation}

For our system, the total energy is given by when we sums over all neighbouring (up, down, left, right) pairs of spins ($<k,l>$), without double-counting

\begin{equation} \label{eq:tot_energy}
    \mathbf{- J E(\mathbf{s}) = \sum\limits_{<k,l>}^Ns_ks_l}
\end{equation}

where $\mathbf{J}$ are the $\mathit{coupling constant}$ and $\mathbf{s}$ represents the spin state of the entire system \textit{microstate} or \textit{spin configuration} and is given by 

\begin{equation}
    \mathbf{s} = \left( s_1, s_2, ... , s_N \right).
\end{equation}

If we sum over all the spin in the system we'll get the total magnetization:

\begin{equation}
    \mathbf{M} (\mathbf{s}) = \sum\limits_i^N s_i.
\end{equation}

To compare the results we get for different lattice sizes, we use magnetization per spin ($m$): 

\begin{equation}
    m (\mathbf{s}) = \frac{M(\mathbf{s})}{N},
\end{equation}

and energy per spin ($\epsilon$):

\begin{equation}
    \epsilon (\mathbf{s}) = \frac{\mathbf{E}(\mathbf{s})}{N}
\end{equation}

We use \textit{Boltzmann distribution} to get the probability for the system state, given a system temperature ($T$):

\begin{equation}
    p(\mathbf{s} ; T) = \frac{1}{Z} e^{- \beta \mathbf{E}(\mathbf{s})},
\end{equation}

where $Z$ is the \textit{partition function} which ia a normalization constant. $\beta$ represents the inverse temperature:

\begin{equation}
    \beta = \frac{1}{k_B T},
\end{equation}
here $k_B$ are called the \textit{Boltzmann constant}.

We also have the partition function ($Z$) summing over all possible $\textbf{s}$ (i):

\begin{equation}
    Z = \sum\limits_{i} e^{- \beta \mathbf{E}(\mathbf{s})},
\end{equation}

which can be expressed analytical as:

\begin{align*}
    Z %&= \sum\limits_{i} e^{- \beta \mathbf{E}(\mathbf{s})} \\
    &= e^{8 \beta J} + 4 + 2 e^{-8 \beta J} + 4 + 4 + e^{-8 \beta J} \\
    &= 2e^{8 \beta J} + 2e^{-8 \beta J} + 12 \\
    &= 4 \cosh (8 \beta J) + 12
\end{align*}

We notice that $<E> = \sum\limits_{s}$

\begin{align*}
    <E> &= \sum\limits_{s} E(s) p(s;1)\\
    &= \sum\limits_{s} \frac{E(s)}{Z} e^{- \beta E(s)} \\
    &= - \frac{1}{Z} \sum\limits_{s} \frac{\partial}{\partial \beta} e^{- \beta E(s)} \\
    &= - \frac{1}{Z} \frac{\partial}{\partial \beta} \sum\limits_{s} e^{- \beta E(s)} \\
    &= - \frac{1}{Z} \frac{\partial}{\partial \beta} Z \\
    &= -8 \frac{J \sinh (8 \beta J)}{\cosh (8 \beta J) + 3}
\end{align*}

Hence

\begin{equation}
    <\epsilon> = <\frac{E}{N}> = -2 \frac{J \sinh (8 \beta J)}{\cosh (8 \beta J) + 3}
\end{equation}

which gives us

\begin{equation}
    <\epsilon^2> = \frac{1}{N^2 Z} \frac{\partial^2}{\partial \beta^2} Z = -4 J^2 \frac{J \sinh (8 \beta J)}{\cosh (8 \beta J) + 3}
\end{equation}

For $<|m|>$ we get

\begin{align*}
    <|m|> = \frac{<|M|>}{N} &= \frac{1}{N Z} \sum\limits_{s} |M(s)| e^{- \beta E(s)} \\
    &= \frac{1}{4 Z} \left( 4 e^{8 \beta J} + 8 + 8 + 4 e^{8 \beta J} \right) \\
    &= \frac{2 e^{8 \beta J}  + 4}{4 \cosh (8 \beta J) + 12} \\
    &= \frac{e^{8 \beta J}  + 2}{2 \cosh (8 \beta J) + 6}
\end{align*}

and 

\begin{align*}
    <m^2> = \frac{<M^2>}{N^2} &= \frac{1}{N^2 Z} \sum\limits_{s} M(s)^2 e^{-8 \beta E(s)} \\
    &= \frac{1}{16 Z} \left( 16 e^{8 \beta J} + 16 + 16 + 16 e^{8 \beta J} \right) \\
    &= \frac{e^{8 \beta J}  + 1}{2 \cosh (8 \beta J) + 6}
\end{align*}

We get the specific heat capacity, also known as normalized to number of spins, by:

\begin{equation}
    E_{\mathbf{s}} = E_{\text{ext}} + \left( -J \sum\limits_{j = 1}^4 s_i s_j \right)
\end{equation}

\begin{align*}
    \Delta E &= E_{\text{after}} - E_{\text{before}} \\
    &= E_{\text{ext}} - J \sum\limits_{j = 1}^4 s_{\text{after}} s_j - E_{\text{ext}} + J \sum\limits_{j = 1}^4 s_{\text{before}} s_j \\
    &= J \left( \left( - \text{after} + s_{\text{before}} \right) 2 \sum\limits_{j = 1}^4 s_j \right)
\end{align*}

since $s_{\text{after}} = - s_{\text{before}}$, we get:

\begin{equation}
    \Delta E = 2 s_{\text{before}} \sum\limits_{j = 1}^4 s_j
\end{equation}

Mapping gives:

\begin{equation}
    \frac{\Delta E}{4} + 2
\end{equation}

TODO: missing anal for the susceptibility (normalized to number of spins)


% ===========================================
\subsection*{The algorithm ?}

% ===========================================
\section{Results and Discussion}\label{sec:results}

\begin{figure}[H]
	\centering
	\includegraphics[scale=0.5]{../figs/testE.pdf} %Imports the figure.
	\caption{The expectation value of $\epsilon$ for $\log_{10} x$ Monte Carlo cycles compared to the analytical solution for $L=2$.}
	\label{fig:testE}
\end{figure}

\begin{figure}[H]
	\centering
	\includegraphics[scale=0.5]{../figs/testM.pdf} %Imports the figure.
	\caption{The expectation value of $|m|$ for $\log_{10} x$ Monte Carlo cycles compared to the analytical solution for $L=2$.}
	\label{fig:testM}
\end{figure}

\begin{figure}[H]
	\centering
	\includegraphics[scale=0.5]{../figs/testCv.pdf} %Imports the figure.
	\caption{The expectation value of $C_v$ for $\log_{10} x$ Monte Carlo cycles compared to the analytical solution for $L=2$.}
	\label{fig:testCv}
\end{figure}

\begin{figure}[H]
	\centering
	\includegraphics[scale=0.5]{../figs/testX.pdf} %Imports the figure.
	\caption{The expectation value of $\chi$ for $\log_{10} x$ Monte Carlo cycles compared to the analytical solution for $L=2$.}
	\label{fig:testX}
\end{figure}

\begin{figure}[H]
	\centering
	\includegraphics[scale=0.5]{../figs/burn_inE.pdf} %Imports the figure.
	\caption{The expectation value of $\epsilon$ for $\log_{10} x$ Monte Carlo cycles for $T=1$ and $T=2.4$ with $L=20$.}
	\label{fig:burn_inE}
\end{figure}

\begin{figure}[H]
	\centering
	\includegraphics[scale=0.5]{../figs/burn_in|m|.pdf} %Imports the figure.
	\caption{The expectation value of $|m|$ for $\log_{10} x$ Monte Carlo cycles for $T=1$ and $T=2.4$ with $L=20$.}
	\label{fig:burn_in|m|}
\end{figure}

\begin{figure}[H]
	\centering
	\includegraphics[scale=0.5]{../figs/hist_T1.pdf} %Imports the figure.
	\caption{Histogram of $10^6$ $\epsilon$ samples for $L=20$ and $T=1.0$}
	\label{fig:hist_T1}
\end{figure}

\begin{figure}[H]
	\centering
	\includegraphics[scale=0.5]{../figs/hist_T2.pdf} %Imports the figure.
	\caption{Histogram of $10^6$ $\epsilon$ samples for $L=20$ and $T=2.4$}
	\label{fig:hist_T2}
\end{figure}

\begin{figure}[H]
	\centering
	\includegraphics[scale=0.5]{../figs/<e>_differing_L.pdf} %Imports the figure.
	\caption{plot of $\langle\epsilon\rangle$ in the range $T \in [2.1,2.4]$ for $L=40,60,80,100$, each run consisiting of $2*10^5$ Monte Carlo cycles}
	\label{fig:e_differing}
\end{figure}

\begin{figure}[H]
	\centering
	\includegraphics[scale=0.5]{../figs/<|m|>_differing_L.pdf} %Imports the figure.
	\caption{plot of $\langle|m|\rangle$ in the range $T \in [2.1,2.4]$ for $L=40,60,80,100$, each run consisiting of $2*10^5$ Monte Carlo cycles}
	\label{fig:m_differing}
\end{figure}

\begin{figure}[H]
	\centering
	\includegraphics[scale=0.5]{../figs/Cv_differing_L.pdf} %Imports the figure.
	\caption{plot of $C_v$ in the range $T \in [2.1,2.4]$ for $L=40,60,80,100$, each run consisiting of $2*10^5$ Monte Carlo cycles}
	\label{fig:Cv_diff_l}
\end{figure}

\begin{figure}[H]
	\centering
	\includegraphics[scale=0.5]{../figs/X_differing_L.pdf} %Imports the figure.
	\caption{plot of $\chi$ in the range $T \in [2.1,2.4]$ for $L=40,60,80,100$, each run consisiting of $2*10^5$ Monte Carlo cycles}
	\label{fig:X_diff_l}
\end{figure}
%

 
% ===========================================
\section{Conclusion}\label{sec:conclusion}



%\bibliographystyle{apalike}
\bibliography{ref}

\onecolumngrid

\newpage

\appendix

\section{GitHub repository}

\centering
\url{https://github.com/sverrewn/FYS3150/tree/project4/project4} 

\section{Possible states} \label{eq_motion}

\begin{table}[H]
    \centering
    \begin{tabular}{|c|c|l|c|c|}
    \hline
        \multicolumn{1}{|l|}{\textbf{Spins in +1}} & \textbf{Sprin configuration} & $\mathbf{- J E(\vec{s}) = \sum\limits_{<k,l>}^Ns_ks_l}$ & \multicolumn{1}{l|}{$\mathbf{M(\vec{s}) = \sum\limits_i^Ns_i}$} & \multicolumn{1}{l|}{\textbf{Degeneracy}} \\ \hline
        4 & 
        
        \begin{tabular}{ll}
            $\uparrow$ & $\uparrow$ \\ 
            $\uparrow$ & $\uparrow$ \\
        \end{tabular}
        
        & $-J(2+2+2+2) = -8J$ & 4 & 1 \\ \hline
        3 & 
        
        \begin{tabular}{ll}
            $\uparrow$ & $\downarrow$ \\ 
            $\uparrow$ & $\uparrow$ \\
        \end{tabular} 
        
        & $-J(0 -2 + 2 + 0) = 0$ & 2 & 4 \\ \hline
        2 & 
        
        \begin{tabular}{ll}
            $\downarrow$ & $\downarrow$ \\ 
            $\uparrow$ & $\uparrow$ \\
        \end{tabular}
        
        & $-J(0 + 0 + 0 + 0) = 0$ & 0 & 4 \\ \hline
        2 & 
        
        \begin{tabular}{ll}
            $\uparrow$ & $\downarrow$ \\ 
            $\downarrow$ & $\uparrow$ \\
        \end{tabular} 
        
        & $-J(-2-2-2-2)= 8J$ & 0 & 2 \\ \hline
        1 & 
        
        \begin{tabular}{ll}
            $\uparrow$ & $\downarrow$ \\ 
            $\downarrow$ & $\downarrow$ \\
        \end{tabular}
        
        & $-J(-2+0 + 0 + 2) = 0$ & -2 & 4 \\ \hline
        0 & 
        
        \begin{tabular}{ll}
            $\downarrow$ & $\downarrow$ \\ 
            $\downarrow$ & $\downarrow$ \\
        \end{tabular}
        
        & $-J(2+2+2+2) = -8J$ & -4 & 1 \\ \hline
    \end{tabular}
    \caption{All possible states of a  $2 \times 2$ lattice with periodic boundary conditions.}
    \label{tab:possible_states}
\end{table}
    

\end{document}