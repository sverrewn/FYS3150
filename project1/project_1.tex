\documentclass[english,notitlepage]{revtex4-1}  % defines the basic parameters of the document
%For preview: skriv i terminal: latexmk -pdf -pvc filnavn



% if you want a single-column, remove reprint

% allows special characters (including æøå)
\usepackage[utf8]{inputenc}
%\usepackage[english]{babel}

%% note that you may need to download some of these packages manually, it depends on your setup.
%% I recommend downloading TeXMaker, because it includes a large library of the most common packages.

\usepackage{physics,amssymb}  % mathematical symbols (physics imports amsmath)
\include{amsmath}
\usepackage{graphicx}         % include graphics such as plots
\usepackage{xcolor}           % set colors
\usepackage{hyperref}         % automagic cross-referencing (this is GODLIKE)
\usepackage{listings}         % display code
\usepackage{subfigure}        % imports a lot of cool and useful figure commands
\usepackage{float}
%\usepackage[section]{placeins}
\usepackage{algorithm}
\usepackage[noend]{algpseudocode}
\usepackage{subfigure}
\usepackage{tikz}
\usetikzlibrary{quantikz}
% defines the color of hyperref objects
% Blending two colors:  blue!80!black  =  80% blue and 20% black
\hypersetup{ % this is just my personal choice, feel free to change things
    colorlinks,
    linkcolor={red!50!black},
    citecolor={blue!50!black},
    urlcolor={blue!80!black}}

%% Defines the style of the programming listing
%% This is actually my personal template, go ahead and change stuff if you want



%% USEFUL LINKS:
%%
%%   UiO LaTeX guides:        https://www.mn.uio.no/ifi/tjenester/it/hjelp/latex/
%%   mathematics:             https://en.wikibooks.org/wiki/LaTeX/Mathematics

%%   PHYSICS !                https://mirror.hmc.edu/ctan/macros/latex/contrib/physics/physics.pdf

%%   the basics of Tikz:       https://en.wikibooks.org/wiki/LaTeX/PGF/Tikz
%%   all the colors!:          https://en.wikibooks.org/wiki/LaTeX/Colors
%%   how to draw tables:       https://en.wikibooks.org/wiki/LaTeX/Tables
%%   code listing styles:      https://en.wikibooks.org/wiki/LaTeX/Source_Code_Listings
%%   \includegraphics          https://en.wikibooks.org/wiki/LaTeX/Importing_Graphics
%%   learn more about figures  https://en.wikibooks.org/wiki/LaTeX/Floats,_Figures_and_Captions
%%   automagic bibliography:   https://en.wikibooks.org/wiki/LaTeX/Bibliography_Management  (this one is kinda difficult the first time)
%%   REVTeX Guide:             http://www.physics.csbsju.edu/370/papers/Journal_Style_Manuals/auguide4-1.pdf
%%
%%   (this document is of class "revtex4-1", the REVTeX Guide explains how the class works)


%% CREATING THE .pdf FILE USING LINUX IN THE TERMINAL
%%
%% [terminal]$ pdflatex template.tex
%%
%% Run the command twice, always.
%% If you want to use \footnote, you need to run these commands (IN THIS SPECIFIC ORDER)
%%
%% [terminal]$ pdflatex template.tex
%% [terminal]$ bibtex template
%% [terminal]$ pdflatex template.tex
%% [terminal]$ pdflatex template.tex
%%
%% Don't ask me why, I don't know.

\begin{document}

\title{FYS3150 - project 1}      % self-explanatory
\author{Sverre Wehn Noremsaune \& Frida Marie Engøy Westby}          % self-explanatory
\date{\today}                             % self-explanatory
\noaffiliation                            % ignore this, but keep it.


\maketitle 

\begin{center}
	\url{https://github.uio.no/comPhys/FYS3150/tree/project1}
\end{center}

    
\section*{Problem 1}

%\subsection*{Problem a}
We have the one-dimensional Poisson equation
\begin{equation}\label{eq:one-dimensional Poisson}
    - \frac{d^2u}{dx^2} = f(x)
\end{equation}
where $f(x)$ is known to be $100e^{-10x}$. We also assume $x \in [0,1]$, that the boundary condition are $u(0) = 0 = u(1)$ and $u(x)$ is

\begin{equation}\label{eq:exact solution one-dimensional Poisson}
    u(x) = 1 - (1 - e^{-10})x-e^{-10x}
\end{equation}
where $u(x)$ is an exact solution to Eq. (1). We can check this analytically by differentiating $u(x)$ twice.
\begin{equation*}
    \begin{split}
    	-u''(x) &= f(x) \\
    	u''(x) &= -f(x) \\
        u(x)' &= 10x^{-10x} - 1 + \frac{1}{e} \\
        u''(x) &= -100e^{-10x}= -f(x) \quad \blacksquare
    \end{split}
\end{equation*}


%\subsection*{Problem b}
%Write a solution for problem 1b here.
\section*{Problem 2}
	See FIG 1. wich shows the plot of the exact Poisson solution. See 'poisson\_exact.cpp' and 'algorithms.cpp' in the github repository, for source code.
	\begin{figure}[H]
		\centering
		\includegraphics[scale=0.55]{plots/poisson_exact.pdf} %Imports the figure.
		\caption{A plot of the exact solution for the Poisson equation from \ref{eq:one-dimensional Poisson} for $x \in[0,1]$}
		\label{fig:Exact poisson}
	\end{figure}


\section*{problem 3}
	We are discretizing the Poisson equation from \ref{eq:one-dimensional Poisson}.

	Discretizing x and setting up some notation:
	\begin{equation*}
		\begin{split}
			x &\rightarrow x_i \\
			u(x) &\rightarrow u_i \\
			i &= 0,1,\dots,n \\
			h &= \frac{x_{max} - x_{min}}{n} \\
			x_i &= x_0 + ih 
		\end{split}
	\end{equation*}

Next we're using the three-point formula to find the second derivative: \\

\begin{equation*}
 	\frac{du^2}{dx^2} = u'' = \frac{u_{i-1} -2 u_i + u_{i+1}}{h^2} + O(h^2)
\end{equation*}

\begin{equation*}
	f_i = -\left(\frac{u_{i-1} - 2u_i + u_{i+1}}{h^2} + O(h^2)\right)
\end{equation*}

Then we approximate and change the notation, $v_i \approx u_i$ and get

\begin{equation}\label{eq:discretized Poisson}
	f_i = \frac{-v_{i-1} +2 v_i - v_{i+1}}{h^2} \\
\end{equation}


\section*{Problem 4}
The equation we got in \ref{eq:discretized Poisson} isn't the most ergonomic for setting up a matrix equation, so we can rewrite it to
\begin{equation}\label{eq:discretized Poisson 2}
	-v_{i-1} + 2v_i - v_{i+1} = h^2f_i
\end{equation}
Equation \ref{eq:discretized Poisson 2} is a set of equations for every i
$$
\begin{matrix}
	i=1 \quad & -v_0 & 2v_1 & -v_2  &       & = h^2f_i \\
	i=2 \quad &      & -v_1 &  2v_2 & - v_3 & = h^2f_2 \\
\end{matrix}
$$
and so on and so forth. We can see that $v_0$ and $v_n$ will end up and alone on their columns, and since we know what they are we simply move them over. \\
$$
\begin{matrix}
	i=1 \quad  & 2v_1 & -v_2  &       & \quad \quad = h^2f_1 + v_0 \\
	i=2 \quad  & -v_1 &  2v_2 & - v_3 & = h^2f_2 \\
\end{matrix}
$$
This can then be easily rewritten as a matrix equation $A\vec{v} = \vec{g}$
$$
\begin{bmatrix}
	2  & -1 & 0  & 0  & \dots & 0 & 0 & \\
	-1 &  2 & -1 & 0  & \dots & 0 & 0 \\
	0  & -1 & 2  & -1 & \dots & 0 & 0 \\
	\vdots & \vdots & \ddots & \ddots & \ddots & \vdots & \vdots \\
	0 & 0 & \dots &-1 & 2 & -1 & 0 \\
	0 & 0 & \dots & 0 & -1 & 2  & -1 \\
	0 & 0 & \dots & 0 & 0 & -1 & 2
	
\end{bmatrix}
\begin{bmatrix}
	v_1 \\ v_2 \\ v_3 \\ \vdots \\ v_{n-3} \\v_{n-2} \\ v_{n-1}
\end{bmatrix}
=
\begin{bmatrix}
	g_1 \\ g_2 \\ g_3 \\ \vdots \\ g{n-3} \\ g{n-2} \\ g_{n-1}
\end{bmatrix}
$$
where $g_i$ is $h^2f_i$ ($+v_0$ for $f_1$ and $+v_n$ for $f_{n-1}$).


\section*{Problem 5}
\subsection*{Problem a}
We can see that n is related to m aince we're "dropping" 2 columns, which gives $n = m-2$

\subsection*{Problem b}
We will find $\vec{{v_i^*}}$ for $1..(n-1)$, meaning everything but the boundary points.


\subsection*{problem 6}
Here will we look at a general tridiagonal matrix, and not the Poisson equation from earlier.

\subsection*{Problem a}
\begin{algorithm}[H]
	\caption{Algorithm for solving general tridiagonal matrix}
	\begin{algorithmic}
		\State arrays a, b, c, u, f, temp of length n
		\\
		\State btemp = b[1]
		\State u[1] = f[1]/btemp
		\\
		\For{ i = 2,3,...,n }
			\State temp[i] = c[i-1] / btemp
			\State btemp = b[i] - a[i] * temp[i]
			\State u[i] = (f[i] - a[i] * u[i-1]) / btemp
		\EndFor
		\For{i = n-1, n-2, ..., 1}
			\State u[i] -= temp[i+1] * u[i+1]
		\EndFor
	\end{algorithmic}
\end{algorithm}

\subsection*{Problem b}
When we analyse the algorithm above, we can see that we get $1 + 6(n-1) + 2(n-1) = 1 + 8(n-1) = 8n - 7$ FLOPs.


\section*{problem 7}
In this problem and the problems below, will we once again look at the Poisson Equation.

\subsection*{Problem a}
See 'general\_tridiag.cpp' and 'algorithms.cpp' in the github repository for the program that solves the Poisson equation with the general algorithm.

\subsection*{Problem b}
	\begin{figure}[H]
	\centering
	\includegraphics[scale=0.55]{plots/gen_tri_cmp_exact.pdf} %Imports the figure.
	\caption{Exact vs numerical comparison for the solution of equation \ref{eq:one-dimensional Poisson}}
	\label{fig:Exact vs approx Poisson }
	\end{figure}


\section*{problem 8}

	\subsection*{Problem a}
	See FIG 3 for the polt showing the (logarithm of) the absolute error.
		\begin{figure}[H]
			\centering
			\includegraphics[scale=0.55]{plots/absolute_error.pdf} %Imports the figure.
			\caption{The absolute error for the general tridiagonal alogrithm in $log_{10}$}
			\label{fig:absolute_error}
		\end{figure}

	\subsection*{Problem b}
	See FIG 4 for the polt showing the (logarithm of) the relative error.
		\begin{figure}[H]
			\centering
			\includegraphics[scale=0.55]{plots/relative_error.pdf} %Imports the figure.
			\caption{The relative error of the general tridiagonal algorithm}
			\label{fig:relative_error}
		\end{figure}

	\subsection*{Problem c}
	See FIG 5 for the plot for the table showing the maximum relative error. We see that for highter $n$ the closer we get to 1.0, and from $n = 1 000 000$ the max relative error becomes 1.0.
		\begin{figure}[H]
			\centering
			\includegraphics[scale=0.75]{plots/biggest_rel_err.pdf} %Imports the figure.
			\caption{Table with the biggest relative error for each iteration}
			\label{fig:biggest_rel_err}
		\end{figure}


\section*{problem 9}
	Since the values of $a,b,c$ never changes, we can just replace the arrays with a constant, saving us from repeatedly calculating $- (-1)\cdot{something}$.
	\subsection*{Problem a}
		\begin{algorithm}[H]
			\caption{Algorithm for solving special tridiagonal matrix}
			\begin{algorithmic}
				\State arrays u, f, temp of length n
				\\
				\State btemp = 2
				\State u[1] = f[1]/ 2
				\\
				\For{ i = 2,3,...,n }
				\State temp[i] = -1 / btemp
				\State btemp = b[i] + temp[i]
				\State u[i] = (f[i] + u[i-1]) / btemp
				\EndFor
				\For{i = n-1, n-2, ..., 1}
				\State u[i] -= temp[i+1] * u[i+1]
				\EndFor
			\end{algorithmic}
		\end{algorithm}
	\subsection*{Problem b}
		FLOPs: $1 + 4(n-1) + 2(n-1) = 1 + 6(n-1) = 6n - 5$
	\subsection*{Problem c}
	See the function 'special\_tridiag' in 'data\_algorithms.cpp' in the github repository for the special algorithm.



\section*{problem 10}
	\begin{figure}[H]
		\centering
		\includegraphics[scale=0.75]{plots/cmp_gen_spec.pdf} %Imports the figure.
		\caption{A table showing the running time of the general and special algorithm for a given $n$}
		\label{fig:cmp_gen_spec}
	\end{figure}


\section*{problem 11}
LU decomposition has a complexity of $O(N^3) (+O(N^2))$ vs our algorithm that runs in $O(N)$. for $n=10^5$ I would expect to: \\
\begin{enumerate}
	\item Run out of memory, since you need to store $8*N*N = 8*10^5*10^5 = 8*10^{10} \approx 80$GB.
	\item To be very slow. The alogrithm's $O$ factor is two orders of magnitude larger, so probably around 100 times as long.
\end{enumerate}

\end{document}