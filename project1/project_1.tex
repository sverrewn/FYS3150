\documentclass[english,notitlepage]{revtex4-1}  % defines the basic parameters of the document
%For preview: skriv i terminal: latexmk -pdf -pvc filnavn



% if you want a single-column, remove reprint

% allows special characters (including æøå)
\usepackage[utf8]{inputenc}
%\usepackage[english]{babel}

%% note that you may need to download some of these packages manually, it depends on your setup.
%% I recommend downloading TeXMaker, because it includes a large library of the most common packages.

\usepackage{physics,amssymb}  % mathematical symbols (physics imports amsmath)
\include{amsmath}
\usepackage{graphicx}         % include graphics such as plots
\usepackage{xcolor}           % set colors
\usepackage{hyperref}         % automagic cross-referencing (this is GODLIKE)
\usepackage{listings}         % display code
\usepackage{subfigure}        % imports a lot of cool and useful figure commands
\usepackage{float}
%\usepackage[section]{placeins}
\usepackage{algorithm}
\usepackage[noend]{algpseudocode}
\usepackage{subfigure}
\usepackage{tikz}
\usetikzlibrary{quantikz}
% defines the color of hyperref objects
% Blending two colors:  blue!80!black  =  80% blue and 20% black
\hypersetup{ % this is just my personal choice, feel free to change things
    colorlinks,
    linkcolor={red!50!black},
    citecolor={blue!50!black},
    urlcolor={blue!80!black}}

%% Defines the style of the programming listing
%% This is actually my personal template, go ahead and change stuff if you want



%% USEFUL LINKS:
%%
%%   UiO LaTeX guides:        https://www.mn.uio.no/ifi/tjenester/it/hjelp/latex/
%%   mathematics:             https://en.wikibooks.org/wiki/LaTeX/Mathematics

%%   PHYSICS !                https://mirror.hmc.edu/ctan/macros/latex/contrib/physics/physics.pdf

%%   the basics of Tikz:       https://en.wikibooks.org/wiki/LaTeX/PGF/Tikz
%%   all the colors!:          https://en.wikibooks.org/wiki/LaTeX/Colors
%%   how to draw tables:       https://en.wikibooks.org/wiki/LaTeX/Tables
%%   code listing styles:      https://en.wikibooks.org/wiki/LaTeX/Source_Code_Listings
%%   \includegraphics          https://en.wikibooks.org/wiki/LaTeX/Importing_Graphics
%%   learn more about figures  https://en.wikibooks.org/wiki/LaTeX/Floats,_Figures_and_Captions
%%   automagic bibliography:   https://en.wikibooks.org/wiki/LaTeX/Bibliography_Management  (this one is kinda difficult the first time)
%%   REVTeX Guide:             http://www.physics.csbsju.edu/370/papers/Journal_Style_Manuals/auguide4-1.pdf
%%
%%   (this document is of class "revtex4-1", the REVTeX Guide explains how the class works)


%% CREATING THE .pdf FILE USING LINUX IN THE TERMINAL
%%
%% [terminal]$ pdflatex template.tex
%%
%% Run the command twice, always.
%% If you want to use \footnote, you need to run these commands (IN THIS SPECIFIC ORDER)
%%
%% [terminal]$ pdflatex template.tex
%% [terminal]$ bibtex template
%% [terminal]$ pdflatex template.tex
%% [terminal]$ pdflatex template.tex
%%
%% Don't ask me why, I don't know.

\begin{document}

\title{FYS3150 - project 1}      % self-explanatory
\author{Sverre Wehn Noremsaune \& Frida Marie Engøy Westby}          % self-explanatory
\date{\today}                             % self-explanatory
\noaffiliation                            % ignore this, but keep it.


\maketitle 
    
\textit{\href{<url>}{https://github.uio.no/comPhys/FYS3150/tree/project1}}
    
\section*{Problem 1}

%\subsection*{Problem a}
We have the one-dimensional Poisson equation
\begin{equation}\label{eq:one-dimensional Poisson}
    - \frac{d^2u}{dx^2} = f(x)
\end{equation}
where $f(x)$ is known to be $100e^{-10x}$. We also assume $x \in [0,1]$, that the boundary condition are $u(0) = 0 = u(1)$ and $u(x)$ is

\begin{equation}\label{eq:exact solution one-dimensional Poisson}
    u(x) = 1 - (1 - e^{-10})x-e^{-10x}
\end{equation}
where $u(x)$ is an exact solution to Eq. (1). We can analytically check this by derivateing $u(x)$ twice.
\begin{equation*}
    \begin{split}
        u(x)' &= 10x^{-10x} - 1 + \frac{1}{e} \\
        u''(x) &= -100e^{-10x} = f(x) \;\;\;\; \blacksquare
    \end{split}
\end{equation*}


%\subsection*{Problem b}
%Write a solution for problem 1b here.

$-\frac{d^2u(x)}{dx^2} = f(x)$

$ x \in \left[0, 1 \right] $

$i = 0,1,\dots,n$

$h = \frac{x_{max} - x_{min}}{n} $

$ x \rarr x_i$

$x_i = x_0 + ih$

We're using the three point formula to find the second derivative, $-\frac{u_{i-1} -2 u_i + u_{i+1}}{h^2} = f_i = f(x_i)$ 

$v_i \approx u_i$ $\implies -\frac{v_{i-1} -2 v_i + v_{i+1}}{h^2} = \frac{-v_{i-1} + 2 v_i - v_{i+1}}{h^2} = f_i, f_i = f(x_i)$, for $i \in [1, n-1]$



---



$v_i = f_i$ is a set of equations for every $i$. By multiplying every line with $h^2$ we get this
$$
\begin{matrix}
	-v_0 & 2v_1 & -v_2  &           & = h^2f_i \\
	& -v_1 &  2v_2 & - v_3 & = h^2f_2 \\
\end{matrix}
$$
and so on and so forth. We know $v_0 = v_n = 0$, so we remove those. This can then be easily rewritten as a matrix equation $A\vec{v} = \vec{g}$
$$
\begin{bmatrix}
	2  & -1 & 0  & 0  & \dots & 0 & 0 & \\
	-1 &  2 & -1 & 0  & \dots & 0 & 0 \\
	0  & -1 & 2  & -1 & \dots & 0 & 0 \\
	\vdots & \vdots & \ddots & \ddots & \ddots & \vdots & \vdots \\
	0 & 0 & \dots &-1 & 2 & -1 & 0 \\
	0 & 0 & \dots & 0 & -1 & 2  & -1 \\
	0 & 0 & \dots & 0 & 0 & -1 & 2
	
\end{bmatrix}
\begin{bmatrix}
	v_1 \\ v_2 \\ v_3 \\ \vdots \\ v_{n-3} \\v_{n-2} \\ v_{n-1}
\end{bmatrix}
=
\begin{bmatrix}
	g_1 \\ g_2 \\ g_3 \\ \vdots \\ g{n-3} \\ g{n-2} \\ g_{n-1}
\end{bmatrix}
$$
$g_i$ is $h^2f_i $


Finally, we can list algorithms by using the \texttt{algorithm} environment, as demonstrated here for algorithm \ref{algo:midpoint_rule}.
\begin{algorithm}[H]
    \caption{Some algorithm}\label{algo:midpoint_rule}
    \begin{algorithmic}
        \State Some maths, e.g $f(x) = x^2$.  \Comment{Here's a comment}
        \For{$i = 0, 1, ..., n-1$}
        \State Do something here 
        \EndFor
        \While{Some condition}
        \State Do something more here 
        \EndWhile
        \State Maybe even some more math here, e.g $\int_0^1 f(x) \dd x$
    \end{algorithmic}
\end{algorithm}
   
\end{document}