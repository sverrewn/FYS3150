%% USEFUL LINKS:
%% -------------
%%
%% - UiO LaTeX guides:          https://www.mn.uio.no/ifi/tjenester/it/hjelp/latex/
%% - Mathematics:               https://en.wikibooks.org/wiki/LaTeX/Mathematics
%% - Physics:                   https://ctan.uib.no/macros/latex/contrib/physics/physics.pdf
%% - Basics of Tikz:            https://en.wikibooks.org/wiki/LaTeX/PGF/Tikz
%% - All the colors!            https://en.wikibooks.org/wiki/LaTeX/Colors
%% - How to make tables:        https://en.wikibooks.org/wiki/LaTeX/Tables
%% - Code listing styles:       https://en.wikibooks.org/wiki/LaTeX/Source_Code_Listings
%% - \includegraphics           https://en.wikibooks.org/wiki/LaTeX/Importing_Graphics
%% - Learn more about figures:  https://en.wikibooks.org/wiki/LaTeX/Floats,_Figures_and_Captions
%% - Automagic bibliography:    https://en.wikibooks.org/wiki/LaTeX/Bibliography_Management  (this one is kinda difficult the first time)
%%
%%                              (This document is of class "revtex4-1", the REVTeX Guide explains how the class works)
%%   REVTeX Guide:              http://www.physics.csbsju.edu/370/papers/Journal_Style_Manuals/auguide4-1.pdf
%%
%% COMPILING THE .pdf FILE IN THE LINUX IN THE TERMINAL
%% ----------------------------------------------------
%%
%% [terminal]$ pdflatex report_example.tex
%%
%% Run the command twice, always.
%%
%% When using references, footnotes, etc. you should run the following chain of commands:
%%
%% [terminal]$ pdflatex report_example.tex
%% [terminal]$ bibtex report_example
%% [terminal]$ pdflatex report_example.tex
%% [terminal]$ pdflatex report_example.tex
%%
%% This series of commands can of course be gathered into a single-line command:
%% [terminal]$ pdflatex report_example.tex && bibtex report_example.aux && pdflatex report_example.tex && pdflatex report_example.tex
%%
%% ----------------------------------------------------


\documentclass[english,notitlepage,reprint,nofootinbib]{revtex4-1}  % defines the basic parameters of the document
% For preview: skriv i terminal: latexmk -pdf -pvc filnavn
% If you want a single-column, remove "reprint"

% Allows special characters (including æøå)
\usepackage[utf8]{inputenc}
% \usepackage[english]{babel}

%% Note that you may need to download some of these packages manually, it depends on your setup.
%% I recommend downloading TeXMaker, because it includes a large library of the most common packages.

\usepackage{physics,amssymb}  % mathematical symbols (physics imports amsmath)
\include{amsmath}
\usepackage{graphicx}         % include graphics such as plots
\usepackage{xcolor}           % set colors
\usepackage{hyperref}         % automagic cross-referencing
\usepackage{listings}         % display code
\usepackage{subfigure}        % imports a lot of cool and useful figure commands
% \usepackage{float}
%\usepackage[section]{placeins}
\usepackage{algorithm}
\usepackage[noend]{algpseudocode}
\usepackage{subfigure}
\usepackage{tikz}
\usetikzlibrary{quantikz}
% defines the color of hyperref objects
% Blending two colors:  blue!80!black  =  80% blue and 20% black
\hypersetup{ % this is just my personal choice, feel free to change things
	colorlinks,
	linkcolor={red!50!black},
	citecolor={blue!50!black},
	urlcolor={blue!80!black}}


% ===========================================


\begin{document}
	
	\title{Numerical simulations of the Penning Trap with multiple particles}  % self-explanatory
	\author{Sverre Wehn Noremsaune, Jon Aleksander Prøitz, Marius Torsvoll, Frida Marie Engøy Westby} % self-explanatory
	\date{\today}                             % self-explanatory
	\noaffiliation                            % ignore this, but keep it.
	
	%This is how we create an abstract section.
	\begin{abstract}
		Numerical solutions of the electromagnetic penning trap using numerical methods of differing order. The simulation is developed to handle singular particles as well as a multi particle system of non interacting and interacting particles. The model finds a numerical solution to the penning trap in 3 dimensions, the resulting simulation is thus plotted in a 2 and 3 dimensional environment to study the phase development of the particles. 
	\end{abstract}
	\maketitle
	
	
	% ===========================================
	\section{Introduction}
	%
	This article aims to develop the theoretical and numerical framework for the simulation of a multi particle penning trap environment. Using the numerical methods of RungeKutta4 and Euler-Chromer we develop a simulation capable of handling multiple particle environments in an interacting and non-interacting way. The Euler-Chromer method is used as a verification of the validity of the RK4 method. 
	
	The Penning trap is used in many experimental physical fields and thus its study is essential. Its use varies from practicalities as a storage device for charged particles and as method of studying moving charged particles. Such examples as fission decay products, ions of interest and stable subatomic particles. The study of Penning traps and its effect on charged particles thus becomes an important part of physics. 
	
	Usage of computer simulations allows for the rapid and complex probing of the properties of the electromagnetic trap. This allows us to probe the trap for a multi particle system and the exploration of a dynamic system of interacting system and for non interacting particles. The analytical theoretical framework is developed using trivial physical properties and is thus further implemented in a numerical setting for numerical methods of differing orders. The resulting simulation is then implemented in -----------
		
	% ===========================================
	\section{Methods}\label{sec:methods}

	The development of the theoretical framework requires no further knowledge of physics and mathematics other than that of elementary electrodynamics, Newtons second law of motion and some linear algebra. These three tools allows us to develop a comprehensive and general theoretical analytical framework for the Penning Trap. Thus the development of the framework may be described as following. 

	% metode for anal
	First we need to declare some of the physical properties and laws used in the development of the theoretical framework.

	We have the formulas for the electric field ($\mathbf{E}$) which are given by
	\begin{equation} \label{eq:electric field}
		\mathbf{E} = -\nabla V,
	\end{equation}
	Where $V$ is the electric potential and $\nabla$ denotes the gradient operator.
	
	At a point $\mathbf{r}$, which is set up by the point charges ${q_1, ..., q_n}$, which are distributed at points ${\mathbf{r}_1, ..., \mathbf{r}_n}$
	\begin{equation} \label{eq:electric field2}
		\mathbf{E} = k_e \sum_{j=1}^n q_j \frac{\mathbf{r}-\mathbf{r}_j}{|\mathbf{r}-\mathbf{r}_j|^3}
	\end{equation}
	here $k_e$ is representing Coulomb’s constant.

	The Lorentz force ($\mathbf{F}$) are given by
	\begin{equation} \label{eq:Lorentz force}
		\mathbf{F} = q\mathbf{E} + q\mathbf{v}\times \mathbf{B},
	\end{equation}
	where $\mathbf{F}$ is the sum of all forces exerted on a particle, $\mathbf{v}$ is the velocity of the particle, $q$ is the charge of our test particle, the electrical field is represented by ($\mathbf{E}$) and the magnetic field by ($\mathbf{B}$). Given that $\mathbf{B}$ is a homogeneous magnetic field in the positive $z$-direction, we can write $\mathbf{B}$ as
	\begin{equation} \label{eq:Bfield}
		\mathbf{B} = B_0\hat{e}_z = (0, 0, B_0)
	\end{equation}
	$B_0$ is the strength of the field, and given that the magnetic field actually exists it then becomes obvious that $B_0 > 0$.

	Using Newtons second law of motion to derive the following relations:
	\begin{equation} \label{eq:newtons2}
		m\ddot{\mathbf{r}} = \sum_i \mathbf{F}_i,
	\end{equation}
	$m$ are the particle mass.

	As shown in appendix \ref{eq_motion} the differential equations governing the time evolution of the particle’s position is given by:
	\begin{equation}\label{eq:newtonx}
		\ddot{x} - \omega_0 \dot{y} - \frac{1}{2}\omega_z^2 x = 0, 
	\end{equation}
	\begin{equation} \label{eq:newtony}
		\ddot{y} + \omega_0 \dot{x} - \frac{1}{2}\omega_z^2 y = 0,
	\end{equation}
	\begin{equation} \label{eq:newtonz}
		\ddot{z} + \omega_z^2 z = 0. 
	\end{equation}

	From appendix \ref{eq_motion2} we see that (\ref{eq:newtonx}) and (\ref{eq:newtony}) can be written together as 
	\begin{equation} \label{eq:newton_f}
		\ddot{f} + i \omega_0 \dot{f} - \frac{1}{2} \omega_z^2 f = 0
	\end{equation}

	In appendix 


	This (\ref{eq:newton_f}) formula have a general solution given as
	\begin{equation} \label{eq:gen_sol}
		f(t) = A_+ e^{-i(\omega_+ t + \phi_+)} + A_- e^{-i(\omega_- t + \phi_-)}
	\end{equation}
	here is $\phi_+$ and $\phi_-$ defined as constant phases, $A_+$ and $A_-$ as the positive amplitudes, and 
	\begin{equation*}
		\omega_\pm = \frac{\omega_0 \pm \sqrt{\omega_0^2 - 2\omega_z^2}}{2}.
	\end{equation*}
	We also have $x(t) = \text{Re} f(t)$ and $y(t) = \text{Im} f(t)$ which gives us the physical coordinates for \ref{eq:newton_f}.
	
	To obtain a bounded solution for the movement in the $xy$-plane (or $|f(t)| < \infty$ when $t\to\infty$) we need the constraints $\omega_0$ and $\omega_z$. First lets assume $\omega \pm \in \mathbb{R}$, then we se that
	$\omega_0^2 - 2 \omega_z^2 > 0 \Leftrightarrow \omega_0^2 > 2 \omega z^2$.
	Putting in the values for $\omega$ (\ref{eq:newtonx}) to express this as a constraint that relates the penning trap parameters to the particle properties and we get:
	\begin{equation*}
		\left( \frac{q B_0}{m} \right) ^2 > \left( \frac{2 q v_0}{m d^2} \right)
	\end{equation*}
	which can be written as
	\begin{equation*}
		\frac{q}{m} B_0^2 > \frac{4 v_0}{d^2} 
	\end{equation*}

    If we rewrite (\ref{eq:gen_sol}) we get
    \begin{equation*}
        f(t) = A_+ (\cos \omega_+ t + i \sin \omega_+ t) + A_- (\cos \omega_- t + i \sin \omega_- t)
    \end{equation*}

    Next we can use this to find the particles upper and lower bounds from the distance from the origin in the in the $xy$-plane, given the physical coordinates for the real part
	\begin{equation*} 
		\Re \left( f(t) \right) + A_+ \cos \omega_+ t + A_- \cos \omega_- t
	\end{equation*}
	and for the imaginary part
	\begin{equation*}
		\Im \left( f(t) \right) + A_+ \sin \omega_+ t + A_- \sin \omega_- t
	\end{equation*}
	Then we are getting upper bounds by when $x(t)$ and $y(t)$ are in upper phase:
	\begin{equation*}
		R_+ = A_+ + A_-
	\end{equation*}

	and lower bounds when they are in opposite phase:
	\begin{equation*}
		R_- = | A_+ + A_- |
	\end{equation*}

	To test our numerical solution, we have made a specific analytical solution where we are assuming that we have have a single charged particle with charge $q$, mass $m$ and the following initial conditions:
	$$ x(0) = x_0, \qquad \dot{x}(0) = 0, $$
	$$ y(0) = 0, \qquad \dot{y}(0) = v_0, $$
	$$ z(0) = z_0, \qquad \dot{z}(0) = 0. $$
	As seen in appendix \ref{specific_anal} we see that the specific analytical solutions for $z(t)$ are:
	\begin{equation}
		z(t) = z_0 \cos (\omega_z t).
	\end{equation}
	And that $f(t)$ have the specific solution for the movement in the $xy$-plane as following:
	\begin{align}
		A_+ &= \frac{v_0 + x_0 \omega_-}{\omega_- - \omega_+}\\
		A_- &= - \frac{v_0 + x_0 \omega_+}{\omega_- - \omega_+}
	\end{align}






	% ===========================================
	
	
	% ===========================================
	\section{Results}\label{sec:results}
	%
	
	
	% ===========================================
	\section{Discussion}\label{sec:discussion}
	%
	
	
	% ===========================================
	\section{Conclusion}\label{sec:conclusion}

	\appendix
	\section{GitHub repository}
	\url{https://github.com/sverrewn/FYS3150/tree/project3/project3}

	\section{The equations of motion} \label{eq_motion}
	Using Newton’s second law (\ref{eq:newtons2}), we get:
	\begin{equation*}
		m \ddot{v} = \sum_i \vectorbold{F}_i \Leftrightarrow m
		\begin{bmatrix}
			\vectorbold{\ddot{x}} \\
			\vectorbold{\ddot{y}} \\
			\ddot{z}
		\end{bmatrix}
		= \sum_i
		\begin{bmatrix}
			\vectorbold{F}_x \\
			\vectorbold{F}_y \\
			F_z
		\end{bmatrix}
	\end{equation*}
	From this we get that 
	\begin{equation*}
		m \ddot{x} = \sum_i F_{x,i}
	\end{equation*}
	Using this and the vectorized expression for the magnetic field (\ref{eq:Bfield}), we get the following cross product
	\begin{equation*}
		\begin{bmatrix}
			\dot{x} \\
			\dot{y} \\
			0
		\end{bmatrix}
		\times
		\begin{bmatrix}
			0\\
			0 \\
			B_0
		\end{bmatrix}
		=
		\begin{bmatrix}
			B_0 \dot{y} \\
			B_0 \dot{x} \\
			0
		\end{bmatrix}
	\end{equation*}
	With this result and the formula for Lorentz force (\ref{eq:Lorentz force}) we can move on and find $m \ddot{x}$:
	\begin{align*}
		m \ddot{x} &= F_x \\
		&= q E_x + (q \vectorbold{V} \times B)_x \\
		&= -q \frac{\partial v}{\partial x} + q
		\begin{bmatrix}
			\dot{x} \\
			\dot{y} \\
			0
		\end{bmatrix}
		\times
		\begin{bmatrix}
			0 \\
			0 \\
			B_0
		\end{bmatrix} \\
		&= - q \left( - \frac{V_0}{d^2} x \right) + q B_0 \ddot{y} \\
	\end{align*}
	We now define $\omega_0$ as $\frac{q B_0}{m}$ and $\omega_z^2$ as $\frac{2 q V_0}{md^2}$ and get the following
	\begin{equation} \label{eq:newton_x}
		\ddot{x} - \frac{q B_0}{m} \dot{y} - \frac{q V_0}{m d^2} x = 0 \Leftrightarrow \ddot{x} - \omega_0 \dot{y} - \frac{1}{2} \omega^2_z x = 0
	\end{equation}
	Then we can repeat this for $\ddot{y}$ and $\ddot{z}$ which gives us
	\begin{align*}
		m \ddot{y} &= F_y \\
		&= q E_y + (q \vectorbold{V} \times B)_y \\
		&= - q \frac{\partial v}{\partial y} - q B_o \dot{x}
	\end{align*}
	for $m \ddot{y}$ and 
	\begin{equation} 
		\ddot{y} + \frac{q B_0}{m} \dot{x} - \frac{q V_0}{m d^2} = 0 \Leftrightarrow \ddot{y} - \omega_0 \dot{x} - \frac{1}{2} \omega^2_z y = 0
	\end{equation}
	for $\ddot{y}$.
	\begin{equation*}
		m \ddot{z} = q E_z = -q \frac{\partial v}{\partial z} = - \frac{4 q v_0}{2 d^2} z
	\end{equation*}
	for $m \ddot{z}$ and
	\begin{equation} 
		\ddot{z} + \omega^2_z z = 0
	\end{equation}
	for $\ddot{z}$. We are also setting $q > 0$ here.

	\section{x and y to a single differential equation} \label{eq_motion2}
	We are now introducing a new formula, which connects (\ref{eq:newtonx}) and (\ref{eq:newtony}).
	\begin{equation*}
		f(t) = x(t) + iy(t)
	\end{equation*}
	Then we can rewrite (\ref{eq:newtonx}) and (\ref{eq:newtony}) as
	\begin{align*} 
		m \ddot{x} - \omega_0 \dot{y} - \frac{1}{2} \omega_z^2 x + i(\ddot{y} + \omega_z^2) = 0 \\
		\ddot{x} + i \ddot{y} + i \omega_0 \dot{x} - \omega_0 \dot{y} - \frac{1}{2} \omega_z^2 (x + iy) = 0 \\
		\ddot{f} + i \omega_0 (\dot{x} + i \dot{y}) - \frac{1}{2} \omega_z^2 f = 0 \\
		\ddot{f} + i \omega_0 \dot{f} - \frac{1}{2} \omega_z^2 f = 0
	\end{align*}


	\section{Specific analytical solution} \label{specific_anal}
	We have the following initial conditions:
	$$ x(0) = x_0, \qquad \dot{x}(0) = 0, $$
	$$ y(0) = 0, \qquad \dot{y}(0) = v_0, $$
	$$ z(0) = z_0, \qquad \dot{z}(0) = 0. $$
	With those initial conditions we can easily see that:
	\begin{equation*}
		x(0) = \Re \left( f(0) \right) = A_+ + A_- = x_0
	\end{equation*}
	and
	\begin{equation*}
		y(0) = \Im \left( f(0) \right) = 0
	\end{equation*}

	Next, we can see that if we derivate (\ref{eq:newton_f}) we will get
	\begin{align*}
		\dot{f} &= \frac{d}{dt} \left[ A_+ e^{- \omega_+ t} + A_- e^{- \omega_- t}  \right] \\
		&= -A_+ i \omega_+ e^{-i \omega t} - A_- i \omega e^{-i \omega t}
	\end{align*}

	Using this result, we can see that:

	\begin{equation*}
		\dot{y} = \Im( \dot{f} ) = -A_+ \omega_+ \cos \omega_+ t - A_- \omega_- \cos \omega_- t
	\end{equation*}
	Putting this together with the initial values for $\dot{y} (0)$, and we get
	\begin{equation*}
		\dot{y} (0) = v_0 = A_+ \omega_+ - A_- \omega_- 
	\end{equation*}
	Doing the same for $\dot{x} (0)$ and we get:
	\begin{equation*}
		\dot{x} = \Re ( \dot{f} ) = A_+ \omega_+ \sin \omega_+ t + A_- \omega_- \sin \omega_- t = 0
	\end{equation*}

	We can now see that for the initial values
	\begin{equation} \label{eq:i}
		x(0) = x_0 \Rightarrow A_+ + A_- = x_0
	\end{equation}
	and
	\begin{equation} \label{eq:ii}
		y(0) = v_0 \Rightarrow A_+ \omega_+ - A_- \omega_- = v_0
	\end{equation}
	
	We can change (\ref{eq:i}) and get the following
	\begin{equation} \label{eq:iii}
		A_- = x_0 - A_+
	\end{equation}
	Then is we sets (\ref{eq:iii}) in to (\ref{eq:ii}) we get that one of the specific solution for $f(t)$ is:
	\begin{align} \label{eq:iv}
		-A_+ \omega_+ - x_0 \omega_- + A_+ \omega_- &= v_0 \nonumber \\
		A_+ (\omega_- \omega_+) &= v_0 + x_0 \omega_- \nonumber \\
		A_+ &= \frac{v_0 + x_0 \omega_-}{\omega_- - \omega_+}
	\end{align}
	Next we sets (\ref{eq:iv}) in to (\ref{eq:iii}) which gives us the other specific solution for $f(t)$:
	\begin{align*}
		A_- &= x_0 - \frac{v_0 + x_0 \omega_-}{\omega_- - \omega_+} \\
		&= \frac{x_0 (\omega_- - \omega_+) - v_0 - x_0 \omega_-}{\omega_- - \omega_+} \\
		&= \frac{-x_0 \omega_+ - v_0}{\omega_- - \omega_+} \\
		&= - \frac{v_0 + x_0 \omega_+}{\omega_- - \omega_+}
	\end{align*}

	If we rewrite (\ref{eq:newtonz}) to 
	\begin{equation*}
		z = A \cos (\omega_z z) + B \sin (\omega_z z)
	\end{equation*}
	and look at the initial value for $z(0)$, we see that
	\begin{equation*}
		z(0) = z_0 \Rightarrow A \cos \omega_z \theta + B \sin \omega_z \theta = z_0 \Rightarrow A = z_0.
	\end{equation*}
	If we derivate $z$, we get 
	\begin{equation*}
		\dot{z} = \omega_z (-z_0 \sin \omega_z t + V \cos \omega_z t).
	\end{equation*}
	Which with the initial value gives us:
	\begin{equation*}
		\dot{z}(0) = \omega_z (- z_0 \sin \omega_z \dot 0 + B \cos \omega_z \theta) = w_z B = 0 \Rightarrow B = 0.
	\end{equation*}

	Now we can see that the specific solution for $z(t)$:
	\begin{equation*}
		z(t) = z_0 \cos (\omega_z t)
	\end{equation*}

	\onecolumngrid
	
	%\bibliographystyle{apalike}
	\bibliography{ref}
	
	
\end{document}