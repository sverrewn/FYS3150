%% USEFUL LINKS:
%% -------------
%%
%% - UiO LaTeX guides:          https://www.mn.uio.no/ifi/tjenester/it/hjelp/latex/
%% - Mathematics:               https://en.wikibooks.org/wiki/LaTeX/Mathematics
%% - Physics:                   https://ctan.uib.no/macros/latex/contrib/physics/physics.pdf
%% - Basics of Tikz:            https://en.wikibooks.org/wiki/LaTeX/PGF/Tikz
%% - All the colors!            https://en.wikibooks.org/wiki/LaTeX/Colors
%% - How to make tables:        https://en.wikibooks.org/wiki/LaTeX/Tables
%% - Code listing styles:       https://en.wikibooks.org/wiki/LaTeX/Source_Code_Listings
%% - \includegraphics           https://en.wikibooks.org/wiki/LaTeX/Importing_Graphics
%% - Learn more about figures:  https://en.wikibooks.org/wiki/LaTeX/Floats,_Figures_and_Captions
%% - Automagic bibliography:    https://en.wikibooks.org/wiki/LaTeX/Bibliography_Management  (this one is kinda difficult the first time)
%%
%%                              (This document is of class "revtex4-1", the REVTeX Guide explains how the class works)
%%   REVTeX Guide:              http://www.physics.csbsju.edu/370/papers/Journal_Style_Manuals/auguide4-1.pdf
%%
%% COMPILING THE .pdf FILE IN THE LINUX IN THE TERMINAL
%% ----------------------------------------------------
%%
%% [terminal]$ pdflatex report_example.tex
%%
%% Run the command twice, always.
%%
%% When using references, footnotes, etc. you should run the following chain of commands:
%%
%% [terminal]$ pdflatex report_example.tex
%% [terminal]$ bibtex report_example
%% [terminal]$ pdflatex report_example.tex
%% [terminal]$ pdflatex report_example.tex
%%
%% This series of commands can of course be gathered into a single-line command:
%% [terminal]$ pdflatex report_example.tex && bibtex report_example.aux && pdflatex report_example.tex && pdflatex report_example.tex
%%
%% ----------------------------------------------------


\documentclass[english,notitlepage,reprint,nofootinbib]{revtex4-1}  % defines the basic parameters of the document
% For preview: skriv i terminal: latexmk -pdf -pvc filnavn
% If you want a single-column, remove "reprint"

% Allows special characters (including æøå)
\usepackage[utf8]{inputenc}
% \usepackage[english]{babel}

%% Note that you may need to download some of these packages manually, it depends on your setup.
%% I recommend downloading TeXMaker, because it includes a large library of the most common packages.

\usepackage{physics,amssymb}  % mathematical symbols (physics imports amsmath)
\include{amsmath}
\usepackage{graphicx}         % include graphics such as plots
\usepackage{xcolor}           % set colors
\usepackage{hyperref}         % automagic cross-referencing
\usepackage{listings}         % display code
\usepackage{subfigure}        % imports a lot of cool and useful figure commands
% \usepackage{float}
%\usepackage[section]{placeins}
\usepackage{algorithm}
\usepackage[noend]{algpseudocode}
\usepackage{subfigure}
\usepackage{tikz}
\usetikzlibrary{quantikz}
% defines the color of hyperref objects
% Blending two colors:  blue!80!black  =  80% blue and 20% black
\hypersetup{ % this is just my personal choice, feel free to change things
	colorlinks,
	linkcolor={red!50!black},
	citecolor={blue!50!black},
	urlcolor={blue!80!black}}


% ===========================================


\begin{document}
	
	\title{Numerical simulations of the Penning Trap with multiple particles}  % self-explanatory
	\author{Sverre Wehn Noremsaune, Jon Aleksander Prøitz, Marius Torsvoll, Frida Marie Engøy Westby} % self-explanatory
	\date{\today}                             % self-explanatory
	\noaffiliation                            % ignore this, but keep it.
	
	%This is how we create an abstract section.
	\begin{abstract}
		In this article we will numerically simulate a Penning trap using Runge-Kutta 4. We also implement Euler-Cromer to check that Runge-Kutta 4 is correctly implemented, and use Euler-Cromer to see how much better Runge-Kutta4 is error-wise. Using our Penning trap model we see how many particles remains after simulating a time period with a varying applied angular frequency. We then take a closer look at one of the frequency bands where we lost many particles.
	\end{abstract}
	\maketitle
	
	
	% ===========================================
	\section{Introduction}
	%
	This article aims to develop the theoretical and numerical framework for the simulation of a multi particle penning trap environment. Using the numerical methods of Runge-Kutta 4 and Euler-Chromer we develop a simulation capable of handling multiple particle environments in an interacting and non-interacting way. The Euler-Chromer method is used as a verification of the validity of the RK4 method. 
	
	The Penning trap is used in many experimental physical fields and thus its study is essential. Its use varies from practicalities as a storage device for charged particles and as method of studying moving charged particles. Such examples as fission decay products, ions of interest and stable subatomic particles. The study of Penning traps and its effect on charged particles thus becomes an important part of physics. 
	
	In section \ref{sec:methods} we describe the mathematics and algorithms needed to simulate the trap. The results are presented and discussed in section \ref{sec:results}. We then make a summary of the results in section \ref{sec:conclusion}
	% ===========================================
	\section{Methods}\label{sec:methods}
% metode for anal
First we need to declare some of the physical properties and laws used in the development of the theoretical framework.

We have the formulas for the electric field ($\mathbf{E}$) which are given by
\begin{equation} \label{eq:electric field}
	\mathbf{E} = -\nabla V,
\end{equation}
Where $V$ is the electric potential and $\nabla$ denotes the gradient operator.

At a point $\mathbf{r}$, which is set up by the point charges ${q_1, ..., q_n}$, which are distributed at points ${\mathbf{r}_1, ..., \mathbf{r}_n}$
\begin{equation} \label{eq:electric field2}
	\mathbf{E} = k_e \sum_{j=1}^n q_j \frac{\mathbf{r}-\mathbf{r}_j}{|\mathbf{r}-\mathbf{r}_j|^3}
\end{equation}
here $k_e$ is representing Coulomb’s constant.

The Lorentz force ($\mathbf{F}$) are given by
\begin{equation} \label{eq:Lorentz force}
	\mathbf{F} = q\mathbf{E} + q\mathbf{v}\times \mathbf{B},
\end{equation}
where $\mathbf{F}$ is the sum of all forces exerted on a particle, $\mathbf{v}$ is the velocity of the particle, $q$ is the charge of our test particle, the electrical field is represented by ($\mathbf{E}$) and the magnetic field by ($\mathbf{B}$). Given that $\mathbf{B}$ is a homogeneous magnetic field in the positive $z$-direction, we can write $\mathbf{B}$ as
\begin{equation} \label{eq:Bfield}
	\mathbf{B} = B_0\hat{e}_z = (0, 0, B_0)
\end{equation}
$B_0$ is the strength of the field, and given that the magnetic field actually exists it then becomes obvious that $B_0 > 0$.

Using Newtons second law of motion to derive the following relations:
\begin{equation} \label{eq:newtons2}
	m\ddot{\mathbf{r}} = \sum_i \mathbf{F}_i,
\end{equation}
$m$ are the particle mass.

As shown in appendix \ref{eq_motion} the differential equations governing the time evolution of the particle’s position is given by:
\begin{equation}\label{eq:newtonx}
	\ddot{x} - \omega_0 \dot{y} - \frac{1}{2}\omega_z^2 x = 0, 
\end{equation}
\begin{equation} \label{eq:newtony}
	\ddot{y} + \omega_0 \dot{x} - \frac{1}{2}\omega_z^2 y = 0,
\end{equation}
\begin{equation} \label{eq:newtonz}
	\ddot{z} + \omega_z^2 z = 0. 
\end{equation}

From appendix \ref{eq_motion2} we see that (\ref{eq:newtonx}) and (\ref{eq:newtony}) can be written together as 
\begin{equation} \label{eq:newton_f}
	\ddot{f} + i \omega_0 \dot{f} - \frac{1}{2} \omega_z^2 f = 0
\end{equation}

In appendix 


This (\ref{eq:newton_f}) formula have a general solution given as
\begin{equation} \label{eq:gen_sol}
	f(t) = A_+ e^{-i(\omega_+ t + \phi_+)} + A_- e^{-i(\omega_- t + \phi_-)}
\end{equation}
here is $\phi_+$ and $\phi_-$ defined as constant phases, $A_+$ and $A_-$ as the positive amplitudes, and 
\begin{equation*}
	\omega_\pm = \frac{\omega_0 \pm \sqrt{\omega_0^2 - 2\omega_z^2}}{2}.
\end{equation*}
We also have $x(t) = \text{Re} f(t)$ and $y(t) = \text{Im} f(t)$ which gives us the physical coordinates for \ref{eq:newton_f}.

To obtain a bounded solution for the movement in the $xy$-plane (or $|f(t)| < \infty$ when $t\to\infty$) we need the constraints $\omega_0$ and $\omega_z$. First lets assume $\omega \pm \in \mathbb{R}$, then we se that
$\omega_0^2 - 2 \omega_z^2 > 0 \Leftrightarrow \omega_0^2 > 2 \omega z^2$.
Putting in the values for $\omega$ (\ref{eq:newtonx}) to express this as a constraint that relates the penning trap parameters to the particle properties and we get:
\begin{equation*}
	\left( \frac{q B_0}{m} \right) ^2 > \left( \frac{2 q v_0}{m d^2} \right)
\end{equation*}
which can be written as
\begin{equation*}
	\frac{q}{m} B_0^2 > \frac{4 v_0}{d^2} 
\end{equation*}

If we rewrite (\ref{eq:gen_sol}) we get
\begin{equation*}
	f(t) = A_+ (\cos \omega_+ t + i \sin \omega_+ t) + A_- (\cos \omega_- t + i \sin \omega_- t)
\end{equation*}

Next we can use this to find the particles upper and lower bounds from the distance from the origin in the in the $xy$-plane, given the physical coordinates for the real part
\begin{equation*} 
	\Re \left( f(t) \right) + A_+ \cos \omega_+ t + A_- \cos \omega_- t
\end{equation*}
and for the imaginary part
\begin{equation*}
	\Im \left( f(t) \right) + A_+ \sin \omega_+ t + A_- \sin \omega_- t
\end{equation*}
Then we are getting upper bounds by when $x(t)$ and $y(t)$ are in upper phase:
\begin{equation*}
	R_+ = A_+ + A_-
\end{equation*}

and lower bounds when they are in opposite phase:
\begin{equation*}
	R_- = | A_+ + A_- |
\end{equation*}

To test our numerical solution, we have made a specific analytical solution where we are assuming that we have have a single charged particle with charge $q$, mass $m$ and the following initial conditions:
$$ x(0) = x_0, \qquad \dot{x}(0) = 0, $$
$$ y(0) = 0, \qquad \dot{y}(0) = v_0, $$
$$ z(0) = z_0, \qquad \dot{z}(0) = 0. $$
As seen in appendix \ref{specific_anal} we see that the specific analytical solutions for $z(t)$ are:
\begin{equation}
	z(t) = z_0 \cos (\omega_z t).
\end{equation}
And that $f(t)$ have the specific solution for the movement in the $xy$-plane as following:
\begin{align}
	A_+ &= \frac{v_0 + x_0 \omega_-}{\omega_- - \omega_+}\\
	A_- &= - \frac{v_0 + x_0 \omega_+}{\omega_- - \omega_+}
\end{align}
	
	
	% ===========================================
\subsection*{Algorithms}
%
The numerical integration method Runge-Kutta\cite{runge-kutta4} 4 is given in algorithm~\ref{algo:rk4}. The algorithm calculates a weighted average using multiple values for the gradient, $k_n$, by using the current position, half a step forward, and a whole step forward, to find the next position. We use 4 points for the weighted average, hence fourth order Runge-Kutta.

%
\begin{figure}
	% NOTE: We only need \begin{figure} ... \end{figure} here because of a compatability issue between the 'revtex4-1' document class and the 'algorithm' environment.
	\begin{algorithm}[H]
		\caption{$4th$ order Runge-Kutta}
		\label{algo:rk4}
		\begin{algorithmic}
			\Procedure{RK4}{$k_1, k_2, k_3, k_4, \Delta t, f, t, y$}
			\State $k_1 \leftarrow \Delta t f(t_i, y_i)$  
			\State $k_2 \leftarrow \Delta t f(t_i + \frac{1}{2}\Delta t, y_i + \frac{1}{2}k_1)$
			\State $k_3 \leftarrow \Delta t f(t_i + \frac{1}{2}\Delta t, y_i + \frac{1}{2}k_2) $
			\State $k_4 \leftarrow \Delta t f(t_{i+1}, y_i + k_3)$
			\State $y_{i+1} = y_i + \frac{1}{6}[k_1 + 2k_2 + 2k_3 + k_4]$
			\EndProcedure
		\end{algorithmic}
	\end{algorithm}
\end{figure}
Euler-Cromer\cite{euler-cromer} is a numerical integration method for coupled equations, given in algorithm~\ref{algo:ec}. The method calculates the next step for the first equations, and uses that result to solve the second equation.
\begin{figure}
	\begin{algorithm}[H]
		\caption{Euler-Cromer}
		\label{algo:ec}
		\begin{algorithmic}
			\Procedure{EC}{$v, x, \Delta t, f$}
			\State $v_{i+1} \leftarrow v_i + \Delta t f(t_i, y_i)$
			\State $x_{i+1} \leftarrow x_i + \Delta t v_{i+1}$
			\EndProcedure
		\end{algorithmic}
	\end{algorithm}
\end{figure}

% ===========================================
\section{Results and discussion}\label{sec:results}
%
To test the Penning trap simulation we firstly plot the $z$ values of a single particle simulated for $100 \mu s$ shown in fig\ref{fig:100us-z}. We have $\frac{2 \pi}{\mu z} \approx 9.05$Hz which corresponds very well with the figure.

\begin{figure}[H]
	\centering
	\includegraphics[scale=0.5]{../figs/one_particle_100us_z.pdf} %Imports the figure.
	\caption{The motion of a single particle in the z-direction in the default Penning Trap configuration over a time period of 100 $\mu$s}
	\label{fig:100us-z}
\end{figure}

Secondly we simulate two particles with and without interaction with the same initial conditions and plot the $xy$-coordinates in fig \ref{fig:particles_xy}. We can clearly see that particle interactions has effect, but the general orbit remains the same.

\begin{figure}[H]
	\centering
	\includegraphics[scale=0.5]{../figs/two_particles_100us.pdf} %Imports the figure.
	\caption{The motion of a two particles in the $xy$ plane with and without particles interaction in the default Penning trap configuration}
	\label{fig:particles_xy}
\end{figure}

Thirdly we make phase space plots for $(x, v_x), (y, v_y), (z, v_z)$ as seen in fig \ref{fig:phase_y} and fig \ref{fig:phase_n}. Particle interactions makes the trajectories more messy, but the general pattern still holds.

\begin{figure}[H]
	\centering
	\includegraphics[scale=0.5]{../figs/phase_space_interact.pdf} %Imports the figure.
	\caption{Phase space plot for $(x, v_x), (y, v_y), (z, v_z)$ in the default Penning Trap configuration with particle interactions}
	\label{fig:phase_y}
\end{figure}
\begin{figure}[H]
	\centering
	\includegraphics[scale=0.5]{../figs/phase_space_no_interact.pdf}
	\caption{Phase space plot for $(x, v_x), (y, v_y), (z, v_z)$ in the default Penning Trap configuration without particle interactions}
	\label{fig:phase_n}
\end{figure}

fourth we make a $3d$ of the the same two particles with and without interaction in fig \ref{fig:3di} and fig \ref{fig:3dni}. This builds up under the previous results of the patterns mostly being the same, but the one with interactions being a bit messier.

\begin{figure}[h!]
	\centering
	\includegraphics[scale=0.5]{../figs/3d_plot_interaction.pdf} %Imports the figure.
	\caption{The motion of a two particles in default Penning Trap configuration with particle interaction}
	\label{fig:3di}
\end{figure}

\begin{figure}[h!]
	\centering
	\includegraphics[scale=0.5]{../figs/3d_plot_no_interaction.pdf} %Imports the figure.
	\caption{The motion of a single particle in the default Penning Trap configuration without particle interaction}
	\label{fig:3dni}
\end{figure}

fifth we compare the numerical approximation with the analytical solution for $5$ different values of $dt$ for both Runge-Kutta 4 and Euler-Cromer in fig \ref{fig:rel_ec} and fig \ref{fig:rel_rk4}. From this we can seek that Runge-Kutta 4 is indeed a good choice, and our implementation is correct. It is strange that $dt=1$ seems to have higher precision than all of the smaller $dt$s from about $40 \mu s$ in, and it might simply be from a numerical error.

\begin{figure}[h!]
	\centering
	\includegraphics[scale=0.5]{../figs/relative_error_EC.pdf} %Imports the figure.
	\caption{The relative error of EC vs the analytical solution}
	\label{fig:rel_ec}
\end{figure}

\begin{figure}[h!]
	\centering
	\includegraphics[scale=0.5]{../figs/relative_error_RK4.pdf} %Imports the figure.
	\caption{The relative error of RK4 vs the analytical solution}
	\label{fig:rel_rk4}
\end{figure}

Finally we calculate the error convergence for EC and RK4 for the five different step sizes in fig


We now simulate our Penning trap with 100 particles and look at how many remain after $500 \mu s$. As fig \ref{fig:part_remain} shows, we have significant particle loss around in a band around approx. $0.44$Hz. This is probably due to the electric field oscillating in harmony with the particles' rotation, causing constructive interference, eventually leading to the particle escaping the trap. With a greater amplitude the resonance is enhanced, causing more particles to leave in a larger band.

\begin{figure}[h!]
	\centering
	\includegraphics[scale=0.5]{../figs/remaining_particles.pdf} %Imports the figure.
	\caption{The number of particles remaining as a function of the applied angular frequency $\omega_V$ in the special case Penning trap}
	\label{fig:part_remain}
\end{figure}

We then narrow in on the band where we lost the most particles in fig \ref{fig:narrow_i} and fig \ref{fig:narrow_ni}. Here we see that particle interactions seems to cause more particles to remain in the trap than without, possibly due to the interaction slightly altering orbits, causing a slight disturbance in the resonance.

\begin{figure}[h!]
	\centering
	\includegraphics[scale=0.5]{../figs/remaining_particles_narrow_interaction.pdf} %Imports the figure.
	\caption{The number of particles remaining as a function of the applied angular frequency $\omega_V$ with particle interaction, zoomed in on the big dip seen in fig \ref{fig:part_remain} in the special case Penning trap}
	\label{fig:narrow_i}
\end{figure}

\begin{figure}[h!]
	\centering
	\includegraphics[scale=0.5]{../figs/remaining_particles_narrow_no_interaction.pdf} %Imports the figure.
	\caption{The number of particles remaining as a function of the applied angular frequency $\omega_V$ without particle interaction, zoomed in around the big dip seen in fig \ref {fig:part_remain}}
	\label{fig:narrow_ni}
\end{figure}
	
	
	% ===========================================
	\section{Conclusion}\label{sec:conclusion}
	We have investigated a numerically simulated Penning trap. First we compared results with and without particle interaction and found that particle interaction doesn't affect the general orbit patterns that much, but will cause slight disturbances. Furthermore we have uncovered a resonance frequency where most particles will escape the trap, and finally we take a closer look at the resonance and investigate what effect particle interactions has here. 
	
	\appendix
	\section{GitHub repository}
	\url{https://github.com/sverrewn/FYS3150/tree/project3/project3}
	
	\section{The equations of motion} \label{eq_motion}
	Using Newton’s second law (\ref{eq:newtons2}), we get:
	\begin{equation*}
		m \ddot{v} = \sum_i \vectorbold{F}_i \Leftrightarrow m
		\begin{bmatrix}
			\vectorbold{\ddot{x}} \\
			\vectorbold{\ddot{y}} \\
			\ddot{z}
		\end{bmatrix}
		= \sum_i
		\begin{bmatrix}
			\vectorbold{F}_x \\
			\vectorbold{F}_y \\
			F_z
		\end{bmatrix}
	\end{equation*}
	From this we get that 
	\begin{equation*}
		m \ddot{x} = \sum_i F_{x,i}
	\end{equation*}
	Using this and the vectorized expression for the magnetic field (\ref{eq:Bfield}), we get the following cross product
	\begin{equation*}
		\begin{bmatrix}
			\dot{x} \\
			\dot{y} \\
			0
		\end{bmatrix}
		\times
		\begin{bmatrix}
			0\\
			0 \\
			B_0
		\end{bmatrix}
		=
		\begin{bmatrix}
			B_0 \dot{y} \\
			B_0 \dot{x} \\
			0
		\end{bmatrix}
	\end{equation*}
	With this result and the formula for Lorentz force (\ref{eq:Lorentz force}) we can move on and find $m \ddot{x}$:
	\begin{align*}
		m \ddot{x} &= F_x \\
		&= q E_x + (q \vectorbold{V} \times B)_x \\
		&= -q \frac{\partial v}{\partial x} + q
		\begin{bmatrix}
			\dot{x} \\
			\dot{y} \\
			0
		\end{bmatrix}
		\times
		\begin{bmatrix}
			0 \\
			0 \\
			B_0
		\end{bmatrix} \\
		&= - q \left( - \frac{V_0}{d^2} x \right) + q B_0 \ddot{y} \\
	\end{align*}
	We now define $\omega_0$ as $\frac{q B_0}{m}$ and $\omega_z^2$ as $\frac{2 q V_0}{md^2}$ and get the following
	\begin{equation} \label{eq:newton_x}
		\ddot{x} - \frac{q B_0}{m} \dot{y} - \frac{q V_0}{m d^2} x = 0 \Leftrightarrow \ddot{x} - \omega_0 \dot{y} - \frac{1}{2} \omega^2_z x = 0
	\end{equation}
	Then we can repeat this for $\ddot{y}$ and $\ddot{z}$ which gives us
	\begin{align*}
		m \ddot{y} &= F_y \\
		&= q E_y + (q \vectorbold{V} \times B)_y \\
		&= - q \frac{\partial v}{\partial y} - q B_o \dot{x}
	\end{align*}
	for $m \ddot{y}$ and 
	\begin{equation} 
		\ddot{y} + \frac{q B_0}{m} \dot{x} - \frac{q V_0}{m d^2} = 0 \Leftrightarrow \ddot{y} - \omega_0 \dot{x} - \frac{1}{2} \omega^2_z y = 0
	\end{equation}
	for $\ddot{y}$.
	\begin{equation*}
		m \ddot{z} = q E_z = -q \frac{\partial v}{\partial z} = - \frac{4 q v_0}{2 d^2} z
	\end{equation*}
	for $m \ddot{z}$ and
	\begin{equation} 
		\ddot{z} + \omega^2_z z = 0
	\end{equation}
	for $\ddot{z}$. We are also setting $q > 0$ here.
	
	\section{x and y to a single differential equation} \label{eq_motion2}
	We are now introducing a new formula, which connects (\ref{eq:newtonx}) and (\ref{eq:newtony}).
	\begin{equation*}
		f(t) = x(t) + iy(t)
	\end{equation*}
	Then we can rewrite (\ref{eq:newtonx}) and (\ref{eq:newtony}) as
	\begin{align*} 
		m \ddot{x} - \omega_0 \dot{y} - \frac{1}{2} \omega_z^2 x + i(\ddot{y} + \omega_z^2) = 0 \\
		\ddot{x} + i \ddot{y} + i \omega_0 \dot{x} - \omega_0 \dot{y} - \frac{1}{2} \omega_z^2 (x + iy) = 0 \\
		\ddot{f} + i \omega_0 (\dot{x} + i \dot{y}) - \frac{1}{2} \omega_z^2 f = 0 \\
		\ddot{f} + i \omega_0 \dot{f} - \frac{1}{2} \omega_z^2 f = 0
	\end{align*}
	
	
	\section{Specific analytical solution} \label{specific_anal}
	We have the following initial conditions:
	$$ x(0) = x_0, \qquad \dot{x}(0) = 0, $$
	$$ y(0) = 0, \qquad \dot{y}(0) = v_0, $$
	$$ z(0) = z_0, \qquad \dot{z}(0) = 0. $$
	With those initial conditions we can easily see that:
	\begin{equation*}
		x(0) = \Re \left( f(0) \right) = A_+ + A_- = x_0
	\end{equation*}
	and
	\begin{equation*}
		y(0) = \Im \left( f(0) \right) = 0
	\end{equation*}
	
	Next, we can see that if we derivate (\ref{eq:newton_f}) we will get
	\begin{align*}
		\dot{f} &= \frac{d}{dt} \left[ A_+ e^{- \omega_+ t} + A_- e^{- \omega_- t}  \right] \\
		&= -A_+ i \omega_+ e^{-i \omega t} - A_- i \omega e^{-i \omega t}
	\end{align*}
	
	Using this result, we can see that:
	
	\begin{equation*}
		\dot{y} = \Im( \dot{f} ) = -A_+ \omega_+ \cos \omega_+ t - A_- \omega_- \cos \omega_- t
	\end{equation*}
	Putting this together with the initial values for $\dot{y} (0)$, and we get
	\begin{equation*}
		\dot{y} (0) = v_0 = A_+ \omega_+ - A_- \omega_- 
	\end{equation*}
	Doing the same for $\dot{x} (0)$ and we get:
	\begin{equation*}
		\dot{x} = \Re ( \dot{f} ) = A_+ \omega_+ \sin \omega_+ t + A_- \omega_- \sin \omega_- t = 0
	\end{equation*}
	
	We can now see that for the initial values
	\begin{equation} \label{eq:i}
		x(0) = x_0 \Rightarrow A_+ + A_- = x_0
	\end{equation}
	and
	\begin{equation} \label{eq:ii}
		y(0) = v_0 \Rightarrow A_+ \omega_+ - A_- \omega_- = v_0
	\end{equation}
	
	We can change (\ref{eq:i}) and get the following
	\begin{equation} \label{eq:iii}
		A_- = x_0 - A_+
	\end{equation}
	Then is we sets (\ref{eq:iii}) in to (\ref{eq:ii}) we get that one of the specific solution for $f(t)$ is:
	\begin{align} \label{eq:iv}
		-A_+ \omega_+ - x_0 \omega_- + A_+ \omega_- &= v_0 \nonumber \\
		A_+ (\omega_- \omega_+) &= v_0 + x_0 \omega_- \nonumber \\
		A_+ &= \frac{v_0 + x_0 \omega_-}{\omega_- - \omega_+}
	\end{align}
	Next we sets (\ref{eq:iv}) in to (\ref{eq:iii}) which gives us the other specific solution for $f(t)$:
	\begin{align*}
		A_- &= x_0 - \frac{v_0 + x_0 \omega_-}{\omega_- - \omega_+} \\
		&= \frac{x_0 (\omega_- - \omega_+) - v_0 - x_0 \omega_-}{\omega_- - \omega_+} \\
		&= \frac{-x_0 \omega_+ - v_0}{\omega_- - \omega_+} \\
		&= - \frac{v_0 + x_0 \omega_+}{\omega_- - \omega_+}
	\end{align*}
	
	If we rewrite (\ref{eq:newtonz}) to 
	\begin{equation*}
		z = A \cos (\omega_z z) + B \sin (\omega_z z)
	\end{equation*}
	and look at the initial value for $z(0)$, we see that
	\begin{equation*}
		z(0) = z_0 \Rightarrow A \cos \omega_z \theta + B \sin \omega_z \theta = z_0 \Rightarrow A = z_0.
	\end{equation*}
	If we derivate $z$, we get 
	\begin{equation*}
		\dot{z} = \omega_z (-z_0 \sin \omega_z t + V \cos \omega_z t).
	\end{equation*}
	Which with the initial value gives us:
	\begin{equation*}
		\dot{z}(0) = \omega_z (- z_0 \sin \omega_z \dot 0 + B \cos \omega_z \theta) = w_z B = 0 \Rightarrow B = 0.
	\end{equation*}
	
	Now we can see that the specific solution for $z(t)$:
	\begin{equation*}
		z(t) = z_0 \cos (\omega_z t)
	\end{equation*}
	\onecolumngrid
	
	%\bibliographystyle{apalike}
	\bibliography{ref}
	
	
\end{document} 